We emphasize the use of model checking over theorem proving, as advocated by \citet{DBLP:conf/kr/HalpernV91}. We use the same ideas as in symbolic model checking, as defined for temporal logics \cite{DBLP:conf/lics/BurchCMDH90}, adapted to DEL, as explained in papers by a subset of the contributors of this demo~\cite{DBLP:conf/atal/CharrierS17,DBLP:conf/aiml/CharrierS18}. Our model checking procedure relies now on symbolic Kripke models, aimed at representing succinctly Kripke models. A Kripke model is a graph whose nodes are \emph{possible worlds}, edges are labeled by agents and an edge $w \rightarrow^a u$ means that agent $a$ considers world $u$ as possible in world $w$. Each world $w$ is equipped with a valuation telling the true atomic propositions in $w$. 
The tool shows that graph on the right of the screen (in Figure~\ref{figure:gui}, the Kripke model has two possible worlds, $w$ and $u$; $p$ is true in $w$ but not in $u$; $\rightarrow_a$ is given in red and $\rightarrow_b$ in blue). 


\newcommand{\succinctsetworlds}{\chi}
\newcommand{\succinctrelation}[1]{\pi_{#1}}
 A symbolic model gives a Boolean formula $\succinctsetworlds(\vec x)$ that succinctly  describes the set of possible worlds: a world is a valuation over Boolean variables $\vec x$ satisfing $\succinctsetworlds(\vec x)$. It also gives, for each agent $a$, a Boolean formula $\succinctrelation a(\vec x, \vec x')$ that tells whether there is an edge labeled by agent $a$ from a world described by a valuation over $\vec x$ and  a world described by a valuation over $\vec x'$. All these Boolean formula are then classically converted in BDDs. %is in a description of the model in a more succinct language. Such representations already exist for other formalisms, such as boolean circuits for searching for Hamiltonian paths in graphs \cite{papadimitriou2003computational}, or BDDs in symbolic model checking .
%
%In the demonstration we use the symbolic description of \cite{DBLP:conf/aiml/CharrierS18} for dynamic epistemic logic \footnote{Which is actually a rewriting of the description of \cite{DBLP:conf/atal/CharrierS17}}, and use the reduction to first-order logic of \cite{DBLP:conf/tableaux/CharrierPS17} to implement this approach with BDDs.
%
Typically, for Hanabi, $\succinctsetworlds(\vec x)$ tells that $\vec x$ describes an initial possible configuration. Formula $\succinctrelation a(\vec x, \vec x')$ tells that agents other than $a$ have the same cards in $\vec x$ and $\vec x'$ (this models the fact that agent $a$ only sees his/her own cards).
%Take for instance any card game. The initial model is described by BDDs for the following boolean formulas: formula $\chi_W$  whose valuations are the possible configurations and formula $\chi_R$ to describe which configurations agents do not distinguish. The formula $\chi_W$ actually describes legal configurations according to the rules of the game, and $\chi_W$ reflect the fact that agents do not see the hand of other agents, so they may imagine as possible any possible dealing.

Dynamic epistemic logic also provides so-called \emph{event models} for describing actions (public announcements, public actions, private announcements/actions, etc.). The reader may refer to the textbook on DEL \cite{DitmarschvdHoekKooi} and to \citet{DBLP:conf/atal/CharrierS17} for symbolic event models, that we do not detail here.
