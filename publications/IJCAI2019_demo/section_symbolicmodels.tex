Symbolic Kripke models aim at representing succinctly so-called pointed Kripke models. A pointed Kripke model is a graph whose nodes are \emph{possible worlds}, edges are labeled by agents and an edge $w \rightarrow^a u$ means that agent $a$ considers world $u$ as possible in world $w$. Each world $w$ is equipped with a valuation telling which atomic propositions is true in $w$. A special world is called the pointed world and represents the true situation, while the other possible worlds are worlds imagined by the agents.
The tool shows that graph in the right-part of the screen (in the example of Figure~\ref{figure:gui}, the Kripke model has two possible worlds). 


\newcommand{\succinctsetworlds}{\chi}
\newcommand{\succinctrelation}[1]{\pi_{#1}}
 A symbolic model provides a Boolean  in a description of the model in a more succinct language. Such representations already exist for other formalisms, such as boolean circuits for searching for Hamiltonian paths in graphs \cite{papadimitriou2003computational}, or BDDs in sy temporal logics \cite{DBLP:conf/lics/BurchCMDH90}.

In the demonstration we use the symbolic description of \cite{DBLP:conf/aiml/CharrierS18} for dynamic epistemic logic \footnote{Which is actually a rewriting of the description of \cite{DBLP:conf/atal/CharrierS17}}, and use the reduction to first-order logic of \cite{DBLP:conf/tableaux/CharrierPS17} to implement this approach with BDDs.

Take for instance any card game. The initial model is described by BDDs for the following boolean formulas: formula $\chi_W$  whose valuations are the possible configurations and formula $\chi_R$ to describe which configurations agents do not distinguish. The formula $\chi_W$ actually describes legal configurations according to the rules of the game, and $\chi_W$ reflect the fact that agents do not see the hand of other agents, so they may imagine as possible any possible dealing.

Dynamic epistemic logic also provides so-called \emph{event models} for describing actions (public announcements, public actions, private announcements/actions, etc.). The reader may refer to the textbook on DEL \cite{DitmarschvdHoekKooi} and to \cite{DBLP:conf/atal/CharrierS17} for symbolic event models, that we do not detail here.