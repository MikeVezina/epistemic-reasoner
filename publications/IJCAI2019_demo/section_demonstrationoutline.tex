
In the demonstration we show a play of the game Hanabi. 
In this game, each agent has cards with a color and a
number, but cannot see his own hand.
In each turn, the user can play the role of one of the agent: he/she can either give the information to some other agent about a number or a color, or play a card. The goal is to play the cards in increasing order for each color.
During the process, the system keeps track on the knowledge of the agents.
More precisely, the system shows the real world (the real distribution of the cards). When the user clicks on an agent, it displays \emph{some} randomly generated possible worlds for that agent (some possible distributions for him/her), as shown in Figure~\ref{figure:guihanabi}.
%
%
%% expliquer la démo. Je me suis peut etre trompé sur le nombre de cartes ici et si on gère les jetons pour les infos il faut le mettre dans le paragraphe.
%
Note that in this demonstration, we will continue to show examples that were already presented in 2018 \cite{DBLP:conf/ijcai/Schwarzentruber18}: Sally and Anne, muddy children, consecutive numbers, etc.

\begin{figure}
	\begin{center}
		
	\end{center}
	\caption{Screenshot of Hanabi in \emph{Hintikka's World}.\label{figure:guihanabi}}
\end{figure}