
Whereas the first version was written in JavaScript, in order to ease the development, the new version is written in TypeScript and relies on the Angular~7 framework.

\subsection{Binary Decision Diagrams}

As shown by \citet{DBLP:conf/atal/CharrierS17}, the symbolic model checking of DEL is PSPACE-complete, thus is critical. We manipulate sets of worlds as well as relations by means of Binary Decision Diagrams. To this aim, we wrote a JavaScript wrapper of the C library CUDD (Colorado University Decision Diagram Package)~\cite{DBLP:journals/sttt/Somenzi01}: we wrote a thin wrapper in C, then compiled into Web Assembly using Emscripten, in order to be usable from our JavaScript module.

In order to show possible worlds for a given agent $a$ in some world $w$, we first construct the BDD of $\succinctrelation a(descr(w), \vec x')$ where $descr(w)$ are the Boolean values of $\vec x$ corresponding to world $w$. We then count the number of possible valuations $\vec x'$ that make $\succinctrelation a(descr(w), \vec x')$ true (BDDs are an efficient representation for counting valuations satisfying a Boolean formula). If the number of such valuations is small, we show all possible worlds, otherwise we randomly generate valuations for $\vec x'$ that makes $\succinctrelation a(descr(w), \vec x')$ true (we randomly select a branch that leads to the ``true'' leaf in the BDD of $\succinctrelation a(descr(w), \vec x')$).

\subsection{Class Architecture}

Figure \ref{figure:architecture} shows the new architecture of \emph{Hintikka's world}. \texttt{EpistemicModel} is an abstract class, used by the graphical user interface (GUI), that is independent from the concrete example (``Muddy Children'', ``Sally and Anne'', ``Hanabi'', etc.) but also, more interestingly, independent from the representation of the epistemic model itself. In particular, an epistemic model can be an \texttt{ExplicitEpistemicModel} (a graph) or a \texttt{SymbolicEpistemicModel} that relies on BDDs. To obtain a comic strips for a given example, it suffices to implement the method \texttt{draw} of a class that inherits from class \texttt{World}, that tells how a possible world is drawn.


\subsection{Adding New Examples}

Providing new examples is easy. Explicit epistemic models are directly described (sets of nodes and of edges). Symbolic epistemic models are described by a Boolean formula $\succinctsetworlds$, and Boolean formulas for $\succinctrelation{a}$. The system provides a way to easily describe how worlds are displayed in the comic strips.

\begin{figure}
	\begin{center}
		\scalebox{0.78}{
			\begin{tikzpicture}[scale=0.75]
			
			\umlclass[x=0,y=2]{EpistemicModel}{
				
			}{}
			
			\umlclass[x=-4.7,y=2]{Graph}{
			}{
			}
			
%			\umlclass[x=-2,y=-2.5]{World}{
%			}{
%			}

	\umlclass[x=7.2,y=2]{BDD}{
}{
}
			
			\umlclass[x=-3,y=-0]{ExplicitEpistemicModel}{
			}{
			}
			
			\umlclass[x=4,y=-2.4]{Hanabi}{
			}{
			}
			
			\umlclass[x=-3,y=-2.4]{SallyAndAnne}{
			}{
			}
			
			\umlclass[x=4,y=-0]{SymbolicEpistemicModel}{
			}{
			}
			
			
		%	\umlassoc[geometry=--, arg1=, mult1=1, align1=right, arg2=, mult2=*, align2=left]{GUI}{EpistemicModel}
		%	\umlassoc[geometry=--, arg1=, mult1=*, arg2=, mult2=1]{EpistemicModel}{World}
			\umlassoc[geometry=--, arg1=, mult1=, arg2=, mult2=1]{ExplicitEpistemicModel}{SallyAndAnne}
			\umlassoc[geometry=--, arg1=, mult1=, arg2=, mult2=1]{SymbolicEpistemicModel}{Hanabi}
			\umlinherit[geometry=|-]{ExplicitEpistemicModel}{Graph}
			\umlinherit[geometry=|-]{SymbolicEpistemicModel}{EpistemicModel}
			\umlinherit[geometry=|-]{ExplicitEpistemicModel}{EpistemicModel}
			\umlassoc[geometry=-|, arg1=, mult1=*, arg2=, mult2=]{SymbolicEpistemicModel}{BDD}			

			
			%\umlunicompo[geometry=-|, arg=titi, mult=*, pos=1.7, stereo=vector]{D}{C}
			%\umlaggreg[arg=tutu, mult=1, pos=0.8, angle1=30, angle2=60, loopsize=2cm]{D}{D}
			
			\end{tikzpicture}}
	\end{center}
	\caption{New architecture of \emph{Hintikka's world}.\label{figure:architecture}}
\end{figure}
