
Whereas the first version was written in Javascript, the new version is written in TypeScript and Angular 7.

\subsection{Binary decision diagrams}

The symbolic model checking of DEL is PSPACE-complete, thus is critical. We manipulate set of worlds, and relations by means of Binary decision diagrams. That is why, for manipulating the binary decision diagrams, we wrote a wrapup in C of the library CUDD \cite{}, that produces a Web Assembly library.

In order to show possible worlds for a given agent $a$ in some world $w$, we first construct the BDD of $\succinctrelation a(descr(w), \vec x')$ where $descr(w)$ are the Boolean values of $\vec x$ corresponding to world $w$. We then count the number of possible valuations $\vec x'$ than makes for randomly values for $\vec x'$ that makes $\succinctrelation a(descr(w), \vec x')$ true (BDDs are convenient for counting, see TODO). If the number of such valuations is small, we show all the possible worlds, otherwise we randomly generate valuations for $\vec x'$ that lead to a true-leaf in the BDD of $\succinctrelation a(descr(w), \vec x')$ (BDDs are also convenient for generating random valuations).

\subsection{Class Architecture}

Figure \ref{figure:architecture} shows the new architecture of \emph{Hintikka's world}. \texttt{EpistemicModel} is an abstract class used by the graphical user interface (GUI), that is independent from the current runnin example (muddy children, Sally and Anne, Hanabi, etc.) but more interestingly independent from the representation of the epistemic model itself. In particular, an epistemic model can be an \texttt{ExplicitEpistemicModel} (a graph) or a \texttt{SymbolicEpistemicModel} that relies on BDDs, depending on the examples.


\subsection{Adding new examples}

The system is easy to use to provide new examples. Explicit epistemic models are directly described (set of nodes and of edges). Symbolic epistemic models are described by a Boolean formula $\succinctsetworlds$, or Boolean formulas for $\succinctrelation{a}$. The system provides a way to easily describe how worlds are displayed in the comic strips.

\begin{figure}
	\begin{center}
		\scalebox{0.8}{
			\begin{tikzpicture}[scale=0.75]
			
			\umlclass[x=0,y=2]{EpistemicModel}{
				
			}{}
			
			\umlclass[x=-6,y=2]{Graph}{
			}{
			}
			
%			\umlclass[x=-2,y=-2.5]{World}{
%			}{
%			}

	\umlclass[x=7,y=2.5]{BDD}{
}{
}
			
			\umlclass[x=-3,y=-0]{ExplicitEpistemicModel}{
			}{
			}
			
			\umlclass[x=4,y=-2.5]{Hanabi}{
			}{
			}
			
			\umlclass[x=-3,y=-2.5]{SallyAndAnne}{
			}{
			}
			
			\umlclass[x=4,y=-0]{SymbolicEpistemicModel}{
			}{
			}
			
			
		%	\umlassoc[geometry=--, arg1=, mult1=1, align1=right, arg2=, mult2=*, align2=left]{GUI}{EpistemicModel}
		%	\umlassoc[geometry=--, arg1=, mult1=*, arg2=, mult2=1]{EpistemicModel}{World}
			\umlassoc[geometry=--, arg1=, mult1=, arg2=, mult2=1]{ExplicitEpistemicModel}{SallyAndAnne}
			\umlassoc[geometry=--, arg1=, mult1=, arg2=, mult2=1]{SymbolicEpistemicModel}{Hanabi}
			\umlinherit[geometry=|-]{ExplicitEpistemicModel}{Graph}
			\umlinherit[geometry=|-]{SymbolicEpistemicModel}{EpistemicModel}
			\umlinherit[geometry=|-]{ExplicitEpistemicModel}{EpistemicModel}
			\umlassoc[geometry=--, arg1=, mult1=, arg2=, mult2=*]{SymbolicEpistemicModel}{BDD}			

			
			%\umlunicompo[geometry=-|, arg=titi, mult=*, pos=1.7, stereo=vector]{D}{C}
			%\umlaggreg[arg=tutu, mult=1, pos=0.8, angle1=30, angle2=60, loopsize=2cm]{D}{D}
			
			\end{tikzpicture}}
	\end{center}
	\caption{New architecture of \emph{Hintikka's world}.\label{figure:architecture}}
\end{figure}